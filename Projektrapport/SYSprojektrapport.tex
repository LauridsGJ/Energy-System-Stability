\documentclass[a4paper, 11pt,oneside,openany, danish]{memoir} % Starter dokumentet af klassen memoir


%%%%%%%%%%%%%%%%%%%%%%%
% PREAMBLE %			
%%%%%%%%%%%%%%%%%%%%%%%



% Papirstørrelse og margener
\usepackage[paper=a4paper, hmargin=1.1in, vmargin=1.1in]{geometry}

% Font encoding og sprog
\usepackage[T1]{fontenc}				% Output encoding
\usepackage[utf8]{inputenc}				% Input encoding
\usepackage[danish]{babel}				% Sprog (orddeling)
\renewcommand{\danishhyphenmins}{22} 	% bedre orddeling, minimum to tegn før og efter deling
\usepackage{lmodern}  					% gør underscores pænere
\usepackage{microtype} 					% laver micro ændringer i text for at udgå luft og orddeling
\usepackage[bottom]{footmisc}			% Sætter fodnoter i bunden.


%% Forside text
%\usepackage{soul} % lege lege
%\sodef\an{}{0.2em}{.9em plus.6em}{1em plus.1em minus.1em}
%\newcommand\stext[1]{\an{\scshape#1}}

% Fyldetekst (Lorem ipsum)
\usepackage{blindtext}

% Til kodeeksempler
\usepackage{listings}

\lstdefinestyle{customc}{
	belowcaptionskip=1\baselineskip,
	breaklines=true,
	frame=L,
	xleftmargin=\parindent,
	language=C,
	showstringspaces=false,
	basicstyle=\footnotesize\ttfamily,
	keywordstyle=\bfseries\color{green!40!black},
	commentstyle=\itshape\color{purple!40!black},
	identifierstyle=\color{blue},
	stringstyle=\color{orange},
}

\lstdefinestyle{customasm}{
	belowcaptionskip=1\baselineskip,
	frame=L,
	xleftmargin=\parindent,
	language=[x86masm]Assembler,
	basicstyle=\footnotesize\ttfamily,
	commentstyle=\itshape\color{purple!40!black},
}

\lstset{escapechar=@,style=customc}

% Lister
\usepackage{enumitem}
\setlist[description]{leftmargin=\parindent, labelindent=\parindent}

% Tabeller
\usepackage{booktabs}
\usepackage{threeparttable}
\usepackage[tableposition=top]{caption}
\usepackage{tabularx}
\usepackage{multirow}					% For at lave pæne tabeller
\usepackage{hhline}						% For at lave endnu pænere tabller
\newcolumntype{C}{>{\let\newline\\\arraybackslash\hspace{0pt}}X}
\usepackage{float}
%matematik
\usepackage{amsmath,amssymb,mathtools,bm}
\newcommand{\tsub}[1]{_{\textup{#1}}}
\def\doubleunderline#1{\underline{\underline{#1}}}
\usepackage[separate-uncertainty = true,multi-part-units=single]{siunitx}
\usepackage{longtable}

% XColor: Farver
\usepackage[svgnames,dvipsnames,x11names]{xcolor}

% Figurer og floats
\usepackage[]{graphicx}
\graphicspath{{figurer/}}
\usepackage{placeins}
\usepackage{float}			% Muliggoer eksakt placering af floats, f.eks. \begin{figure}[H]

%%% Tegning af kasser
%\usepackage{calc,graphicx,color}
%\definecolor{mygreen}{rgb}{0,0.6,0}
%\definecolor{mygray}{rgb}{0.5,0.5,0.5}

% Biblatex til referencer
\usepackage[backend=bibtex]{biblatex}
\addbibresource{bibfil.bib}





% Hyper ref
\usepackage[ unicode=true, colorlinks=false, linktocpage=true, 
pdfborder={0 0 0}, pdfstartpage=1, pdfstartview=FitV, breaklinks=true,
pdfpagemode=UseNone, pageanchor=true, pdfpagemode=UseOutlines,
plainpages=false, bookmarksnumbered, bookmarksopen=true,
bookmarksopenlevel=1, hypertexnames=true, pdfhighlight=/O, urlcolor=Black,
linkcolor=Black, citecolor=Black]{hyperref}

% Clever ref
\usepackage{cleveref}



\settocdepth{subsection}
\setsecnumdepth{subsection}

% Sidetal
% Sidetal
\let\footruleskip\undefined
\usepackage{fancyhdr}
\usepackage{lastpage}
\pagestyle{fancy} 
\fancyhf{} 

\fancyhead[R]{\leftmark}
\fancyfoot[R]{\thepage \hspace{0.008in} af \pageref{LastPage}}

\fancypagestyle{}{
	\renewcommand{\headrulewidth}{0pt}
	\fancyhf{}
	\fancyfoot[R]{\thepage \hspace{0.008in} af \pageref{LastPage}}%
	
}

% Starten på dokumentet
\begin{document}


%%%%%%%%%%%%%%%%%%%%%%%
		       % FORSIDEN %			
%%%%%%%%%%%%%%%%%%%%%%%
% !TEX root = ../SYSprojektrapport.tex
% SKAL STÅ I TOPPEN AF ALLE FILER FOR AT MASTER-filen KOMPILERES 
\thispagestyle{empty}
	{\centering
	{\scshape\LARGE Aarhus Universitet \par}
	\vspace{1cm}
	{\scshape\Large Elektrisk energiteknologi\par}
	\vspace{0.5cm}
	{\scshape\Large Energy System Stability\par}
	{\scshape\Large Gruppe 1\par}
	{\scshape\Large Projektrapport\par}
	\vspace{1.5cm}
	{\huge\bfseries Implementering af\\ husstandsbatterier\par}
	\vspace{2cm}
	
	\begin{figure}[H] % (alternativt [H])
		\centering
		\includegraphics[width=1\textwidth]{figurer/Forside}
		\label{fig:Forside}
	\end{figure}

	{\Large
	201505115 - Laurids Givskov Jørgensen\\
	13114 - Jeppe Hansen\\   }
	\vfill
	Underviser\par
	Björn Andresen

	\vfill

	{\large \today\par}
\par}

%%%%%%%%%%%%%%%%%%%%%%%

             % RESUME & ABSTRACT %			
             
%%%%%%%%%%%%%%%%%%%%%%%            

\pagebreak

%%%%%%%%%%%%%%%%%%%%%%%


%%%%%%%%%%%%%%%%%%%%%%%
         % INDHOLDSFORTEGNELSE %			
%%%%%%%%%%%%%%%%%%%%%%%
\frontmatter
\tableofcontents

%%%%%%%%%%%%%%%%%%%%%%%
                        % KAPITLER %			
%%%%%%%%%%%%%%%%%%%%%%%                        
\mainmatter
                    
\chapter{Indledning}
% !TEX root = ../SYSprojektrapport.tex
% SKAL STÅ I TOPPEN AF ALLE FILER FOR AT MASTER-filen KOMPILERES 

\label{Indledning}

Denne rapport er udarbejdet i forbindelse med kurset Energy System stability på Aarhus universitet. 
Danmark har et af verdens mest stabile energiforsyningerne med gode forbindelse til omkringliggende lande. Men pga. de høje ambitioner om at nedsætte CO2 udslippet i hele Europa ændres energiproduktionerne i stor grad til vedvarende energikilder. Da de vedvarende energikilder afhængig af vejrforholdene bliver produktionen mere fluktuerende, det kan skabe ubalancen mellem produktion og forbrug og derved give anledning til stabilitetsproblemer. En af mulighederne for at undgå stabilitetsproblemerne er ved at implementer batterier i el nettet.

I dette projekt undersøges hvordan husstandsbatterier kan stabilisere et el net. Det undersøges hvilke former for stabilitetsproblemer batterier kan afhjælpe og derefter opbygges et simplificeret el net i PowerFactory for at verificere de teoretiske undersøgelser. Tilslut dokumenteres resultater og der diskuteres på de enkelte undersøgelser.



\chapter{Problemformulering}
% !TEX root = ../prj4projektrapport.tex
% SKAL STÅ I TOPPEN AF ALLE FILER FOR AT MASTER-filen KOMPILERES 

Danmark er et land med stor kapacitet indenfor vedvarende energikilder, især indenfor vindenergi. Dette gør at der i perioder med gunstige vindforhold kan forekomme overproduktion, som er nødvendig at eksportere. En måde at sikre den grønne energi bliver brugt i Danmark er ved at oplagre energien i batterier. \\
Der vil derfor undersøges muligheden for implementering af batterier i husstande. Det forventes at en stor mængde batterier i husstande vil kunne oplagre overproduktionen af grøn energi. \\


Derudover vil det undersøges om batterierne vil kunne stabilisere det danske elnet ved fejltilstande og udglatte produktionen henover 24 timer, da batterierne vil kunne bidrage med strøm i perioder med stort forbrug. \\
Desuden vil det undersøges om de decentrale hustandsbatterier har en fordel frem for større centrale batteriparker, der er tilsluttet på højere spændingsniveau i elnettet, som f.eks. Tesla’s batteripark i Australien.


\section{Afgrænsning}
% !TEX root = ../SYSprojektrapport.tex
% SKAL STÅ I TOPPEN AF ALLE FILER FOR AT MASTER-filen KOMPILERES 

\label{Afgraensning}

Projektet afgrænses til at indeholde en undersøgelse af følgende fire cases:

\begin{description}
	\item[Case 1] Husstandsbatteriers evne til at stabilisere elnettet ved fejl på nettet.
	\item[Case 2] Husstandsbatteriers evne til at absorbere overproduktion.
	\item[Case 3] Husstandsbatteriers evne til at kompensere for tab af produktion.
	\item[Case 4] Husstandsbatteriers stabilierende effekt af elnettet kontra en central batteripark.
\end{description}	
	
Der vil i projektet ikke laves dynamiske modeller, og der vil derved ikke undersøges hvordan batterier reagere under kortslutninger. Ved fejl undersøges det hvordan systemet reagere f.eks. efter en kortslutning er udkoblet. Casene er beskrevet i kapitel \ref{Simulering}.  
 

I projektet fokuseres der på de stabiliserende fordele det kunne medføre at implementere batterier i elnettet. Det er derfor ikke undersøgt hvordan batterier kan implementeres eller hvordan økonomien er i det. For at simplificere modellen vil der ikke implementeres frekvensregulering på produktionsenhederne.


\chapter{Systemstabilitet}
% !TEX root = ../SYSprojektrapport.tex
% SKAL STÅ I TOPPEN AF ALLE FILER FOR AT MASTER-filen KOMPILERES 

\label{Systemstabilitet}
For at kunne forstå den effekt det vil have at implementere batterier i et elnet, skal man først kende til systemstabilitet og de problemer der er relateret til at sikre et stabilt netværk.

Et elektriske netværk i steady state tilstand skal kunne håndtere forstyrrelse og fejl i nettet, sådan at det ikke fejlramte net forbliver i dets steady state tilstand eller finder et ny steady state arbejdspunkt efter fejlen er clearet.\\
Derved sikres forsyning til de ikke direkte påvirkede dele af nettet. Systemstabilitet er på den måde viden omkring hvordan man kan designe sit netværk, for at undgå blackouts af større dele eller hele det elektriske netværk.

Systemstabilitet opdeles i tre hovedgrupper: Rotorvinkelstabilitet, frekvensstabilitet og spændingsstabilitet.\\
Hver gruppe opdeles i forskellige typer ustabilitet der kan forekomme pga. af forstyrrelse eller fejl i nettet. På figur \ref{fig:Overview} ses et overblik over klassificering af systemstabilitet.
\footnote{https://www.semanticscholar.org/paper/Definition-and-classification-of-power-system-joint-Kundur-Paserba/5d9e9822845e172a7518218073831dab4ad41643}

\begin{figure}[H] % (alternativt [H])
	\centering
	\includegraphics[width=0.9\textwidth]{figurer/Classification_of_power_system_stability}
	\caption{Klassificering af systemstabilitet}
	\label{fig:Overview}
\end{figure}

I dette projekt er det hovedsageligt relevant at undersøge implementeringen af batteriers effekt på nettets frekvensstabilitet og spændingsstabilitet. Dette skyldes at frekvensstabilitet hænger sammen med forholdet mellem produktion og belastning af nettet og spændingsstabilitet hænger sammen med belastningen af nettet, samt kompensering af reaktiv effekt. Rotorvinkelstabilitet er primært relateret til synkron generatorers evne til at forblive synkroniseret med nettet under fejl og vil derfor ikke være et fokus i dette projekt. \\
En forklaring af de stabilitetsproblemer der kan forekomme i forbindelse med frekvensstabilitet og spændingsstabilitet er derfor gennemgået i de følgende kapitler.

\section{Frekvensstabilitet}
% !TEX root = ../SYSprojektrapport.tex
% SKAL STÅ I TOPPEN AF ALLE FILER FOR AT MASTER-filen KOMPILERES 

\label{Frekvensstabilitet}

Frekvensstabilitet dækker over et elektrisk systems evne til at opretholde eller hurtigt genoprette systemfrekvensen, selvom systemet påvirkes af forstyrrelse, der vil resultere i ubalance mellem produktion og belastning. Systemet skal altså kunne reguleres således at der igen opnåes balance mellem produktion og belastning i systemet, uden signifikant tab af belastning.\\
Vedvarende frekvens ustabilitet vil føre til udkobling af produktionsenheder og forbrugere.

Frekvensstabilitet inddeles i \textit{short term} og \textit{long term} stabilitetsproblemer, som vist på figur \ref{fig:Overview}.\\
\textit{Short term} har en varighed på op til 1 minut og defineres som pludselige ændringer i belastningsforholdet. Dette kunne være tab af en større generationsenhed, en transmissionslinje eller en stor forbruger. \textit{Short term} problemer kan udvikle sig til \textit{long term}, hvis systemet, med de umiddelbare til rådige reguleringsreserver, ikke formår at skabe balance mellem produktion og belastning igen.\\
\textit{Long term} har en varighed fra 1 minut til flere timer og defineres som længerevarende afvigelser fra den nominelle systemfrekvens. Et \textit{long term} problem kunne opstå gennem mistiming af reguleringen af et stort synkron kraftværk grundet en forudset ændring i produktionen fra vedvarende energikilder i systemet, som følge af vejrændringer.
Typiske reguleringshastigheder er for et kulkraftværk 1\% i minuttet og for et gaskraftværk 10-15\% i minuttet.

\section{Frekvensregulering og kontrol}
Frekvensstabiliteten opretholdes i normal drift af elnettet gennem handel af elektricitet. Dette sker på timebasis og elektricitetsmarkedet er derfor ansvarlig for at sørge produktionen matcher det forbrug markedet forventer. Ved ubalance i belastningsforholdet har Transmission System Operatoren (TSO) - i Danmark er det Energinet.dk - ansvaret for regulering af produktionen. Der defineres i ENTSO-E Policy 1\footnote{ENTSO-E Policy 1} fire forskellige kontrolreserver til at opretholde den nominelle systemfrekevens.

\begin{description}
	\item[Primær kontrol] Det enkelte kraftværks egen regulering. Kan aktiveres på sekunder.
	\item[Sekundær kontrol] Midlertidige produktionsreserve, styret af TSO'en, der kan aktiveres på sekunder/minutter med en varighed på ca 15 minutter.
	\item[Tertiær kontrol] Manuelt aktiverede produktionsreserve, styret af TSO'en. Anvendt til længerevarende ustabilitet.
	\item[Time kontrol] Handel på energimarkedet overvåges af TSO'en for at forudse behov for regulering af produktionen.
\end{description}

Måden de forskellige kontrolreserver interagerer med hinanden på kan illustres som vist på figur \ref{fig:Frekvenskontrol}. En afvigelse fra systemfrekvensen vil føre til aktivering af den primære kontrol, for at undgå tab af synkrone generationsenheder og stabilisere frekvensen ved et nyt arbejdspunkt tæt på nominel systemfrekvens. Derefter vil den sekundære kontrol aktiveres for at genoprette den nominelle systemfrekvens. Hvis den sekundære kontrol ikke formår at genoprette systemfrekvensen eller hvis generationsenheder er blevet tabt aktiveres den tertiære kontrol. Den tertiære kontrol dækker også over planlagte aktivering/regulering af produktionsenheder, der vil blive anvendt ved tab af større generationsenheder i forbindelse med forstyrrelsen/fejlen.

\begin{figure}[H] % (alternativt [H])
	\centering
	\includegraphics[width=0.9\textwidth]{figurer/Frekvenskontrol}
	\caption{Skematisk overblik over aktiveringen af kontrolreserver til frekvensregulering}
	\label{fig:Frekvenskontrol}
\end{figure}

ENTSO-E Policy 1 nedsætter også nogle krav til reservekapaciteten i det centraleuropæiske elnet. Vigtige krav er at den primære kontrol skal aktiveres ved frekvensafvigelser på $\pm$20mHz og den skal være fuldt ud aktiveret ved afvigelser på $\pm$200mHz. Størrelsen af den primære reserve bliver fastsat årligt og er på 3000MW. Den primære reserve er normeret fordelt på kraftværker i hele det centraleuropæiske elnet.

Den sekundære kontrol implementeres som en Load Frequency Control (LFC) struktur som vist på figur \ref{fig:sekundaerkontrol}. 

\begin{figure}[H]
	\centering
	\includegraphics[width=0.9\textwidth]{figurer/Sekundaer_kontrol}
	\caption{LFC struktur}
	\label{fig:sekundaerkontrol}
\end{figure}

Ændringen i frekvens sammenholdes med ændringen i aktiv effekt gennem en K-faktor der er faktor der beregnes ud fra et område i systemets afvigelse fra systemfrekvensen i forhold til effektregulering pga. aktivering af den primære kontrol. Den samlede påkrævede effektregulering filtreres herefter og hvis afvigelsen i frekvens er udenfor et dødbåndsområde vil en PI regulator regulere den generede effekt, sådan at systemfrekvensen igen kan opnåes.


\section{Batterier som aktivt netelement}

Måden hvorpå batterier i elnettet kan bidrage til at stabilisere systemfrekvensen er at de både kan absorbere og genere effekt afhængigt af behovet og deres opladningstilstand. Dette kan give fordele i et elnet, hvor andelen er vedvarende energikilder er stor og derved har mindre reguleringsreserve i situationer med utilstrækkeligt vejr.

Her kan batterier fungere som både primær og sekundær reserve grundet den hurtige reguleringsmulighed der er i et rent elektrisk system. Et husstandsbatterier har en typisk kapacitet på 14kWh og kan levere 5kW nominelt og 7kW peak\footnote{TESLA}. Derfor vil enkelte husstandsbatterier ikke kunne bidrage særlig meget til balancering af produktion og forbrug, men en samling af mange husstandsbattier - en såkaldt aggrering vil kunne - vil kunne bidrage med betydelig effekt. Dette kombineret med muligheden for at oplade batterierne i perioder med mulighed for stor produktion fra grønne generationsenheder, så deres kapacitet er til rådighed i perioder med lav produktion fra grønne produktionsenheder kan tilføre den nødvendige fleksibilitet til elnettet for at kunne opretholde systemfrekvensen i en elnet med stor andel af vedvarende energikilder.

Placeringen og typen af batterierne forventes for frekvensstabiliteten at være ubetydelig, da den den generede effekt bare skal matche den absorberede for systemet for at opretholde den nominelle systemfrekvens.

Noget med inerti??? eller rettere mangel derpå??




\section{Spændingsstabilitet}
% !TEX root = ../SYSprojektrapport.tex
% SKAL STÅ I TOPPEN AF ALLE FILER FOR AT MASTER-filen KOMPILERES 

\label{Spaendingsstabilitet}

Spændingsstabilitet er et systems evne til at opretholde den nominelle spænding ved busbarere og forbrugere ved normale og unormale forhold i nettet. Spændingen ved belastning må maksimalt svinge med $\pm$10\% iht. INDSÆT STANDARD!, det er derfor vigtig at systemet kan regulere spændingen.  

Spændingsstabilitet er iht. figur \ref{fig:Overview} opdelt i \textit{Large- and Small Disturbance Voltage Stability} og herefter \textit{Short Term} og \textit{Long Term}. Store og små forstyrrelser henviser til hvor omfattende problemet er. \\
\textit{Small Disturbance Voltage Stability} kan være resultatet af et øgede forbrug samtidig med at en linje er ude pga. service, så der vil være et større tab i de resterende linjer, eller hvis en enkelt linje ryger ud pga. en fejl.\\
\textit{Large Disturbance Voltage Stability} er som regel resultatet af flere hændelser i nettet der skaber et spændingsfald i et større område eller i værste fald forårsager blackout.

Begge dele kan være \textit{Short Term} og \textit{Long Term}. Det kommer an på hvor hurtigt systemet kan reguleres, omlægges eller fjerne fejlen i nettet. På figur \ref{fig:VoltageTime} ses hvilke elementer der kan stabilisere systemet på kort og lang sigt. Til hurtigt regulering kan anvendes mindre generatorer, konverterer eller kondensator banke, hvor der over længere tid f.eks. kan laves en regulering i et kraftværker eller opstartes en gas turbine.  

\begin{figure}[H] % (alternativt [H])
	\centering
	\includegraphics[width=0.6\textwidth]{figurer/Voltage_time}
	\caption{Oversigt over Short and Long Term disturbances}
	\label{fig:VoltageTime}
\end{figure}

Hovedsageligt opstår spændingsustabilitet når systemet ikke kan levere nok reaktiv effekt. Når der tilføjes reaktiv effekt skal spændingen stige, men hvis spændingen falder er systemet ustabilt. Spændingsustabilitet kan både forekomme som overspænding og spændingsfald. Overspænding kan ske ved for stor produktion i forhold til belastning, men også hvis der bliver tilføjet for meget reaktiv effekt i systemet pga. for meget kapacitiv belastning fra f.eks. shunt kondensatorer. 

Den største årsag til spændingsustabilitet er spændingsfald der hovedsageligt opstår pga. den induktive reaktans i transmissionslinjer. Tabet i linjerne forøges sammen med belastning, hvilket vil kræve en større mængde reaktiv effekt af systemet. På figur \ref{fig:Voltage1} ses et simpelt system med spændingskilde $E_{s}$, kabel impedans $Z_{LN}$, belastnings impedans $Z_{LD}$. 


\begin{figure}[H] % (alternativt [H])
	\centering
	\includegraphics[width=0.5\textwidth]{figurer/Voltage_system}
	\caption{Skematisk diagram af et net system med forsyning, transmissionslinje og belastning}
	\label{fig:Voltage1}
\end{figure}

Ved at undersøge systemet med en variable belastning ses det at den maksimale effekt overførelse er hvor spændingen ved belastningen $V_{R}$ er lig med spændingsfaldet i transmissionslinjen. Dette kaldes det kritiske punkt. Det er også i dette punkt hvor $Z_{LN}$ og $Z_{LD}$ er lige store. Hvis $Z_{LD}$ er mindre end $Z_{LN}$ stiger strømmen og spændingen bliver mindre, det vil skabe et stort spændingsfald og gøre systemet ustabilt. Det normale arbejdspunkt i et system ligger på ca 80\% af maks. belastningen.

\begin{figure}[H] % (alternativt [H])
	\centering
	\includegraphics[width=0.7\textwidth]{figurer/Voltage_curve}
	\caption{graf for variable belastning i systemet}
	\label{fig:Voltage2}
\end{figure}


\section{Batterier som aktiv netelement}

Da batterier er en fleksibel belastning kan de hjælpe på spændingsstabilitet, både ved overspænding og spændingsfald. Ved normal drift stabiliseres nettet da batterier kan af- og oplades alt efter hvor hårdt nettet er belastet. Ved service af kabler og andet udstyr kan batterier aflaste det resterende net så det ikke belastes så hårdt og ved pludselig fejl kan batterierne hurtigt kobles ind, indtil der bliver omlagt forbindelser i understationerne. Batterierne kan derfor blive en stor hjælp for spændingsstabilitet og kan hjælpe på små og store forstyrrelser og på kort og lang sigt.

\chapter{Model og validering}
% !TEX root = ../SYSprojektrapport.tex
% SKAL STÅ I TOPPEN AF ALLE FILER FOR AT MASTER-filen KOMPILERES 

\label{Modelopbygning}

\section{Model}

For at undersøge husstandsbatterier indflydelse i et el net er der simuleret 5 byer med ca. 2000 hustande, der alle har et batteri installeret. Hver hustand er sat til at have et gennemsnitligt dagligt forbrug på 14kWh,\footnote{https://orsted.dk/Privat/Faa-en-lavere-regning/Kom-godt-i-gang-og-spar-paa-energien/Test-dit-gennemsnitsforbrug/Elforbrug}. Det giver et gennemsnitlig forbrug på 1183kW per by, og en total gennemsnitlig belastning for modelen på 5914kW. For at finde en maks. belastning, forholdet mellem fossile brændsler og vedvarende energi er der taget udgangspunkt i Energinets belastning og produktions information fra Danmark. Det er undersøgt hvor stor en del den gennemsnitlige belastning i modelen udgør af Danmarks gennemsnitlige belastning og derved fundet en skallerings faktor på 716. Denne skallerings faktor er brugt til at finde en samlede maks. belastning for byerne ved dividere med Danmarks maksimale belastning. Den maksimale belastning for hver by ligger derved på 1524kW.

 
 \begin{figure}[H] % (alternativt [H])
 	\centering
 	\includegraphics[width=1\textwidth]{figurer/Sim_model_2}
 	\caption{Systemets belastning og distribution}
 	\label{fig:Simdis}
 \end{figure}
    



\begin{figure}[H] % (alternativt [H])
	\centering
	\includegraphics[width=1\textwidth]{figurer/Sim_model_1}
	\caption{systemets produktion transmission}
	\label{fig:SimTrans}
\end{figure}

% !TEX root = ../SYSprojektrapport.tex
% SKAL STÅ I TOPPEN AF ALLE FILER FOR AT MASTER-filen KOMPILERES 

\label{Validering}
\section{Validering}

For at sikre at PowerFactory modellen af det dansk elnet, er designet som forventet er der blevet gennemført en validering af modellen. Validering er lavet på baggrund af en spændingsfaldsberegninger og en kortslutningsberegninger. I valideringen forsyner den synkrone generatorenhed hele nettet og belastningsforholdet er designet således der er overenstemmelse mellem produktion og belastning. Vindmølleparken er koblet ud ved Transmission central 160kV busbar og alle batterier og solceller er koblet ud ved deres POC. Ydermere er evt. ringforbindelser og redundant forbindelser koblet ud, således at kun Town3 forsynes direkte fra Distribution busbar. Se figur \ref{fig:Simdis} og \ref{fig:SimTrans} for referencer.

\subsection{Validering med spændingsfaldsberegninger}
Spændingsfaldet blev beregnet på 10kV distributionsbusbaren. Dette blev gjort ved at beregne impedansen for alle dele af modellen. Herunder er et beregningseksempel for hvert elnet element i valideringsmodellen.

Synkron generator:
\begin{figure}[H] % (alternativt [H])
	\centering
	\includegraphics[width=0.85\textwidth]{figurer/Synkron_generator_validering}
	\caption{Synkron generator impedans}
	\label{fig:SGimpedans}
\end{figure}

Transformer:
\begin{figure}[H] % (alternativt [H])
	\centering
	\includegraphics[width=0.9\textwidth]{figurer/Transformer_validering}
	\caption{Transformer impedans}
	\label{fig:Trafoimpedans}
\end{figure}

Kabel:
\begin{figure}[H] % (alternativt [H])
	\centering
	\includegraphics[width=0.8\textwidth]{figurer/Kabel_validering}
	\caption{Kabel impedans}
	\label{fig:Kabelimpedans}
\end{figure}

Byer:
\begin{figure}[H] % (alternativt [H])
	\centering
	\includegraphics[width=0.9\textwidth]{figurer/By_validering}
	\caption{By impedans ved simuleringsspænding}
	\label{fig:Byimpedans}
\end{figure}

 Når impedansen for alle elementer i valideringssimuleringen er beregnet kan man beregne spændingsfaldet ved en bestemt busbar med spændingsdeler formlen. 10kV distributionsbusbaren blev brugt til validering. Zsource dækker her over alle impedanser før 10kV distributionsbusbaren og Zload dækker over alle impedanser efter 10kV distributionsbusbaren. VT4HV er 10kV.
 
 \begin{figure}[H] % (alternativt [H])
 	\centering
 	\includegraphics[width=0.6\textwidth]{figurer/Spaendingsfald_validering}
 	\caption{Spændingsfald ved beregning og simulering}
 	\label{fig:Spaendingsfald_validering}
 \end{figure}
 
 Som det ses på figur \ref{fig:Spaendingsfald_validering} er afvigelsen mellem beregning og simulering 46V vinkel 0.003deg, som er en tilladelig afvigelse på et 10kV referencepunkt. Ud fra spændingsfaldsvalidering er modellen dermed accepteret.
 
\subsection{Validering med kortslutningsberegninger}
Kortslutningsberegninger blev også beregnet til 10kV distributionsbusbaren. Dette gøres ved at finde kortslutningsimpedansen der er den samme som Zsource, bortset fra at synkron generatorens kortslutningseffekt er større end dens rated effekt og dermed er ses en mindre impedans for synkron generatoren. Den nye beregning er synkron generator impedansen ses på figur \ref{fig:SGimpedansSC}.

\begin{figure}[H] % (alternativt [H])
	\centering
	\includegraphics[width=0.75\textwidth]{figurer/Synkron_generator_valideringSC}
	\caption{Synkron generator kortslutningsimpedans}
	\label{fig:SGimpedansSC}
\end{figure}

Derefter kan kortslutningsstrømmen ved en trefaset kortslutning på 10kV distributionsbusbaren beregnes, samt findes ved simulering.

\begin{figure}[H] % (alternativt [H])
	\centering
	\includegraphics[width=0.75\textwidth]{figurer/Kortslutningsstroem_validering}
	\caption{Kortslutningsstrøm ved beregning og simulering}
	\label{fig:SCvalidering}
\end{figure}

På figur \ref{fig:SCvalidering} ses det at forskellen på beregning og simulering kun er 4A. Derfor accpeteres modellen også gennem validering med kortslutningsberegninger.



\subsection{Resultat af validering}

Udfra de beskrevne resultater i begge dele af valideringen, er der opnåede acceptable afvigelser mellem beregning og simulering. Derfor accepteres modellen til videre simuleringen af de tidligere nævnte projekt cases.

\chapter{Simulering}
% !TEX root = ../SYSprojektrapport.tex
% SKAL STÅ I TOPPEN AF ALLE FILER FOR AT MASTER-filen KOMPILERES 

\label{Simulering}

\section{Case 1: Husstandsbatteriers evne til at stabilisere elnettet ved fejl på nettet}
\label{SimCase1}
For at simulere case 1 anvendes et scenarie, hvor systemet er fuldt belastet, dvs. hver by trækker 1,524MW med pf 0,95 lagging. Efter 10s mister systemet kablet 10kV Cable Town5 ved en udkobling. Dette betegnes som en small disturbance fejl der vil skabe short term ubalance i powerflowet.\\
Vindmølleparken generer 2MW med pf 0,99 lagging, solcellerne i hver by levere 0,15MW med pf 0.8 lagging og synkron generatoren er reference maskine.\\
Case 1 simuleres i fire forskellige tilstande.

\begin{description}
	\item[Tilstand 1] Alle batterierne er frakoblet.
	\item[Tilstand 2] Batteriet i Town4 leverer 0,25MW og batteriet i Town5 leverer 0,5MW. Begge med pf 0,95 lagging.
	\item[Tilstand 3] Batteriet i Town4 leverer 0,5MW og batteriet i Town5 leverer 1MW. Begge med pf 0,95 lagging.
	\item[Tilstand 4] Batteriet i Town4 leverer 0,75MW og batteriet i Town5 leverer 1,5MW. Begge med pf 0,95 lagging.
\end{description}

Parametrene der overvåges er for synkron generatoren P, Q og V.
For batterierne i Town4 og Town5 P og Q. Samt V for Transmission 60kV busbar, Town1 busbar, Town2 busbar, Town3 busbar, Town4 busbar og Town5 busbar.
Derudover overvåges systemfrekvensen.

\section{Case 2: Husstandsbatteriers evne til at absorbere overproduktion}
\label{SimCase2}
For at simulere case 2 anvendes et scenarie, hvor systemet er fuldt belastet, dvs. hver by trækker 1,524MW med pf 0,95 lagging. Efter 10s mister systemet Town2 pga. en udkobling ved Distribution by2 busbar. Dette betegnes som en large disturbance fejl, der kunne udvikle sig til en long term ubalance i systemet. I case 2 simuleres kun den short term påvirkning udkoblingen vil have på systemet, da det stadigvæk vil give et indblik i batteriernes evne til at absorbere overproduktion.\\
Vindmølleparken generer 2MW med pf 0,95 lagging, solcellerne i hver by levere 0,15MW med pf 0.8 lagging og synkron generatoren er reference maskine.\\
Case 2 simuleres i to forskellige tilstande.

\begin{description}
	\item[Tilstand 1] Alle batterierne er frakoblet.
	\item[Tilstand 2] Batterierne i alle byer kobles ind 0,5s efter fejlen og absorberer 0,304MW som kompensation for tabet af byen. Alle med pf 0,95 lagging.
\end{description}

Parametrene der overvåges er for synkron generatoren P, Q og V.
For batterierne i alle 5 byer P og Q. Samt V for Transmission 60kV busbar, Town1 busbar, Town2 busbar, Town3 busbar, Town4 busbar og Town5 busbar.
Derudover overvåges systemfrekvensen.

\section{Case 3: Husstandsbatteriers evne til at kompensere for tab af produktion}
\label{SimCase3}
For at simulere case 3 anvendes et scenarie, hvor systemet er fuldt belastet, dvs. hver by trækker 1,524MW med pf 0,95 lagging, samt at kablet Transmission cable2 er koblet ud pga. vedligeholdelse. Efter 10s mister systemet vindmølleparken pga. en udkobling ved Transmission central 160kV busbar. Dette betegnes som en large disturbance fejl, der kunne udvikle sig til en long term ubalance eller muligvis blackout i større dele af systemet. I case 3 simuleres kun den short term påvirkning det vil have på systemet, da det stadig vil give et indblik i batteriernes evne til at kompensere for mistet produktion.\\
Vindmølleparken generer 2MW med pf 0,95 lagging, solcellerne i hver by levere 0,15MW med pf 0.8 lagging og synkron generatoren er reference maskine.\\
Case 3 simuleres i fire forskellige tilstande.

\begin{description}
	\item[Tilstand 1] Alle batterierne er frakoblet.
	\item[Tilstand 2] Alle batterier leverer 0,5MW med pf 0,95 lagging.
	\item[Tilstand 3] Alle batterier leverer 1MW med pf 0,95 lagging.
	\item[Tilstand 4] Alle batterier leverer 1,35MW (Byerne kan betegnes som selvforsynende) med pf 0,95 lagging.
\end{description}

Parametrene der overvåges er for synkron generatoren P, Q og V.
For batterierne i alle 5 byer P og Q. Samt V for Transmission 60kV busbar, Town1 busbar, Town2 busbar, Town3 busbar, Town4 busbar og Town5 busbar.
Derudover overvåges systemfrekvensen.

\section{Case 4: Husstandsbatteriers stabiliserende effekt af elnettet kontra en central batteripark}
\label{SimCase4}
Case 4 er den samme som case 3, bortset fra at batteriparken anvendes i stedet for husstandsbatterierne.
Case 4 simuleres i fire forskellige tilstande.

\begin{description}
	\item[Tilstand 1] Batteripark er frakoblet.
	\item[Tilstand 2] Batteriparken leverer 2,5MW med pf 0,95 lagging.
	\item[Tilstand 3] Batteriparken leverer 5MW med pf 0,95 lagging.
	\item[Tilstand 4] Batteriparken leverer 6,75MW med pf 0,95 lagging.
\end{description}

Parametrene der overvåges er for synkron generatoren P, Q og V.
For batteriparken P og Q. Samt V for Transmission 60kV busbar, Town1 busbar, Town2 busbar, Town3 busbar, Town4 busbar og Town5 busbar.
Derudover overvåges systemfrekvensen.


\chapter{Resultat og diskussion}
% !TEX root = ../SYSprojektrapport.tex
% SKAL STÅ I TOPPEN AF ALLE FILER FOR AT MASTER-filen KOMPILERES 

\label{ResultatOgDiskussion}

\section{Case 1: Husstandsbatteriers evne til at stabilisere elnettet ved fejl på nettet}
I dette afsnit præsenteres resultater for simuleringen af case 1 iht. beskrivelsen i afsnit \ref{SimCase1}. I alle fire tilstande er spændingsændringen ved Town5 busbar (rød linje) og Transmission central 60kV busbar (Grøn linje) samt frekvensændringen på Transmission central 60kV busbar (Grøn linje), præsenteret på hhv. spændingsgraf og frekvensgraf. Derudover er der lavet opsamling over spænding samt effektoverførelse andre relevante steder i systemet i tabel \ref{fig:C1Overview}. \\ \\

\textbf{Tilstand 1: Alle batterierne er frakoblet.}
\begin{figure}[H]
	\centering
	\begin{minipage}[b]{0.48\textwidth}
		\centering
		\includegraphics[width=1.00\textwidth]{figurer/SmallDisturbance/Voltage1} % Venstre billede
	\end{minipage}
	\hfill
	\begin{minipage}[b]{0.48\textwidth}
		\centering
		\includegraphics[width=1.00\textwidth]{figurer/SmallDisturbance/Freq1} % Højre billede
	\end{minipage}
	\\ % Figurtekster og labels
	\begin{minipage}[t]{0.48\textwidth}
		\caption{Case 1, Tilstand 1, Spændingsgraf} % Venstre figurtekst og label
		\label{fig:C1T1V}
	\end{minipage}
	\hfill
	\begin{minipage}[t]{0.48\textwidth}
		\caption{Case 1, Tilstand 1, Frekvensgraf} % Højre figurtekst og label
		\label{fig:C1T1F}
	\end{minipage}
\end{figure}

\textbf{Tilstand 2: Batteriet i Town4 leverer 0,25MW og batteriet i Town5 leverer 0,5MW. Begge med pf 0,95 lagging.}
\begin{figure}[H]
	\centering
	\begin{minipage}[b]{0.48\textwidth}
		\centering
		\includegraphics[width=1.00\textwidth]{figurer/SmallDisturbance/Voltage2} % Venstre billede
	\end{minipage}
	\hfill
	\begin{minipage}[b]{0.48\textwidth}
		\centering
		\includegraphics[width=1.00\textwidth]{figurer/SmallDisturbance/Freq2} % Højre billede
	\end{minipage}
	\\ % Figurtekster og labels
	\begin{minipage}[t]{0.48\textwidth}
		\caption{Case 1, Tilstand 2, Spændingsgraf} % Venstre figurtekst og label
		\label{fig:C1T2V}
	\end{minipage}
	\hfill
	\begin{minipage}[t]{0.48\textwidth}
		\caption{Case 1, Tilstand 2, Frekvensgraf} % Højre figurtekst og label
		\label{fig:C1T2F}
	\end{minipage}
\end{figure}

\textbf{Tilstand 3: Batteriet i Town4 leverer 0,5MW og batteriet i Town5 leverer 1MW. Begge med pf 0,95 lagging.}
\begin{figure}[H]
	\centering
	\begin{minipage}[b]{0.48\textwidth}
		\centering
		\includegraphics[width=1.00\textwidth]{figurer/SmallDisturbance/Voltage3} % Venstre billede
	\end{minipage}
	\hfill
	\begin{minipage}[b]{0.48\textwidth}
		\centering
		\includegraphics[width=1.00\textwidth]{figurer/SmallDisturbance/Freq3} % Højre billede
	\end{minipage}
	\\ % Figurtekster og labels
	\begin{minipage}[t]{0.48\textwidth}
		\caption{Case 1, Tilstand 3, Spændingsgraf} % Venstre figurtekst og label
		\label{fig:C1T3V}
	\end{minipage}
	\hfill
	\begin{minipage}[t]{0.48\textwidth}
		\caption{Case 1, Tilstand 3, Frekvensgraf} % Højre figurtekst og label
		\label{fig:C1T3F}
	\end{minipage}
\end{figure}

\textbf{Tilstand 4: Batteriet i Town4 leverer 0,75MW og batteriet i Town5 leverer 1,5MW. Begge med pf 0,95 lagging.}
\begin{figure}[H]
	\centering
	\begin{minipage}[b]{0.48\textwidth}
		\centering
		\includegraphics[width=1.00\textwidth]{figurer/SmallDisturbance/Voltage4} % Venstre billede
	\end{minipage}
	\hfill
	\begin{minipage}[b]{0.48\textwidth}
		\centering
		\includegraphics[width=1.00\textwidth]{figurer/SmallDisturbance/Freq4} % Højre billede
	\end{minipage}
	\\ % Figurtekster og labels
	\begin{minipage}[t]{0.48\textwidth}
		\caption{Case 1, Tilstand 4, Spændingsgraf} % Venstre figurtekst og label
		\label{fig:C1T4V}
	\end{minipage}
	\hfill
	\begin{minipage}[t]{0.48\textwidth}
		\caption{Case 1, Tilstand 4, Frekvensgraf} % Højre figurtekst og label
		\label{fig:C1T4F}
	\end{minipage}
\end{figure}

\textbf{Tilstandsoverblik}
\begin{figure}[H] % (alternativt [H])
	\centering
	\includegraphics[width=1\textwidth]{figurer/SmallDisturbance/Overview}
	\caption{Overblik for spænding og effektoverførelse i nettet}
	\label{fig:C1Overview}
\end{figure}


I de 3 første tilstande observeres et spændingsfald ved Town5 busbar, når linjen 10kV Cable Town5 udkobles. Størrelsen afhænger af batteriernes effektbidrag. I tilstand 4 ses det at spændingen ved Town5 busbar stiger ved udkoblingen. Spænding ved Transmissions central 60kV busbar forbliver konstant på ca. 1pu i alle tilstande. Spændingsfaldet i tilstandende 1 - 3 er forventet, fordi tabet af den redundante linje vil øge kilde impedansen set fra Town5 busbar. Spændingsfaldet bliver mindre desto mere batteri effektbidrag pga. den reduceret strøm i transmissions og distributions kabler. Spændingsstigningen i tilstand 4 kan forklares ved at Town5 har større produktion end forbrug, derved vil den levere effekt til resten af systemet. Ved tab af 10kV Cable Town5, bliver load impedansen set fra Town5 busbar større og der vil opleves en spændingsstigning ved byens POC.

I de 4 tilstande ses det at frekvensen bliver mere stabilt ved større batteri bidrag. Stigningen på frekvensen i de første tilstande sker fordi at den samlede belastning bliver mindre da spændingen falder, produktionen forsøger at nedregulere, men kan ikke regulere hurtigt nok ift. belastningen. Dette sker ikke i tilstand 4 da spændingen ikke falder.   

% !TEX root = ../SYSprojektrapport.tex
% SKAL STÅ I TOPPEN AF ALLE FILER FOR AT MASTER-filen KOMPILERES 

% !TEX root = ../SYSprojektrapport.tex
% SKAL STÅ I TOPPEN AF ALLE FILER FOR AT MASTER-filen KOMPILERES 

\label{ResultatOgDiskussion2}

\section{Case 2: Husstandsbatteriers evne til at absorbere overproduktion}
I dette afsnit præsenteres resultater for simuleringen af case 2 iht. beskrivelsen i afsnit \ref{SimCase1}. I de 2 tilstande er spændingsændringen ved \textit{Town5 busbar} (rød linje) og \textit{Transmission central 60kV busbar} (Grøn linje) samt frekvensændringen på \textit{Transmission central 60kV busbar} (Grøn linje) præsenteret på hhv. spændingsgraf og frekvensgraf. Derudover er der lavet opsamling over spænding samt effektoverførelse andre relevante steder i systemet i tabel \ref{fig:C2Overview}. \\ \\

\textbf{Tilstand 1: Alle batterierne er frakoblet.}
\begin{figure}[H]
	\centering
	\begin{minipage}[b]{0.48\textwidth}
		\centering
		\includegraphics[width=1.00\textwidth]{figurer/LossOfTown/Voltage1} % Venstre billede
	\end{minipage}
	\hfill
	\begin{minipage}[b]{0.48\textwidth}
		\centering
		\includegraphics[width=1.00\textwidth]{figurer/LossOfTown/Freq1} % Højre billede
	\end{minipage}
	\\ % Figurtekster og labels
	\begin{minipage}[t]{0.48\textwidth}
		\caption{Case 2, Tilstand 1, Spændingsgraf} % Venstre figurtekst og label
		\label{fig:C2T1V}
	\end{minipage}
	\hfill
	\begin{minipage}[t]{0.48\textwidth}
		\caption{Case 2, Tilstand 1, Frekvensgraf} % Højre figurtekst og label
		\label{fig:C2T1F}
	\end{minipage}
\end{figure}

\textbf{Tilstand 2: Batterierne i alle byer kobles ind 0,5s efter fejlen og absorberer 0,304MW som kompensation for tabet af byen. Alle med pf 0,95}
\begin{figure}[H]
	\centering
	\begin{minipage}[b]{0.48\textwidth}
		\centering
		\includegraphics[width=1.00\textwidth]{figurer/LossOfTown/Voltage2} % Venstre billede
	\end{minipage}
	\hfill
	\begin{minipage}[b]{0.48\textwidth}
		\centering
		\includegraphics[width=1.00\textwidth]{figurer/LossOfTown/Freq2} % Højre billede
	\end{minipage}
	\\ % Figurtekster og labels
	\begin{minipage}[t]{0.48\textwidth}
		\caption{Case 2, Tilstand 2, Spændingsgraf} % Venstre figurtekst og label
		\label{fig:C2T2V}
	\end{minipage}
	\hfill
	\begin{minipage}[t]{0.48\textwidth}
		\caption{Case 2, Tilstand 2, Frekvensgraf} % Højre figurtekst og label
		\label{fig:C2T2F}
	\end{minipage}
\end{figure}

\textbf{Tilstandsoverblik}
\begin{figure}[H] % (alternativt [H])
	\centering
	\includegraphics[width=1\textwidth]{figurer/LossOfTown/Overview}
	\caption{Overblik for spænding og effektoverførelse i nettet}
	\label{fig:C2Overview}
\end{figure}

I tilstand 1 ses det at spænding stiger betydeligt ved \textit{Town5 busbar} og \textit{Transmissions central 60kV busbar}. I tilstand 2 stiger spændingen kortvarigt indtil batterierne kobles ind. Spænding stiger i tilstand 1, da der er for meget produktion i forhold til belastning. Da batterierne i tilstand 2 påbegynder opladning hurtigt efter tabet af \textit{Town2} kompensere de for den manglende belastning og stabilisere derfor spændingen til normal.

Frekvensen starter med at stige lidt pga. den udkoblede belastning. hvorefter den begynder at falde lidt igen. Dette sker formentlig, fordi at belastning stiger i de ikke afkoblede byer pga. overspændingen og det ser ikke ud til at Powerfactory tilpasser spændingsniveauet efter at belastningen stiger igen. I tilstand 2 stiger frekvensen formentlig pga. det hurtigt afkoblede produktion og da inertien er ret høj i systemet går der lidt tid inden den tilpasser sig systemet igen.
  

% !TEX root = ../SYSprojektrapport.tex
% SKAL STÅ I TOPPEN AF ALLE FILER FOR AT MASTER-filen KOMPILERES 

\section{Case 3: Husstandsbatteriers evne til at kompensere for tab af produktion}
I dette afsnit præsenteres resultater for simuleringen af case 3 iht. beskrivelsen i afsnit \ref{SimCase3}. I alle fire tilstande er spændingsændringen ved Town5 busbar (rød linje) og Transmission central 60kV busbar (Grøn linje) samt frekvensændringen på Transmission central 60kV busbar (Grøn linje), præsenteret på hhv. spændingsgraf og frekvensgraf. Derudover er der lavet opsamling over spænding samt effektoverførelse andre relevante steder i systemet i tabel \ref{fig:C3Overview}. \\ \\

\textbf{Tilstand 1: Alle batterierne er frakoblet.}
\begin{figure}[H]
	\centering
	\begin{minipage}[b]{0.48\textwidth}
		\centering
		\includegraphics[width=1.00\textwidth]{figurer/LargeDisturbance/Voltage1} % Venstre billede
	\end{minipage}
	\hfill
	\begin{minipage}[b]{0.48\textwidth}
		\centering
		\includegraphics[width=1.00\textwidth]{figurer/LargeDisturbance/Freq1} % Højre billede
	\end{minipage}
	\\ % Figurtekster og labels
	\begin{minipage}[t]{0.48\textwidth}
		\caption{Case 2, Tilstand 1, Spændingsgraf} % Venstre figurtekst og label
		\label{fig:C2T1V}
	\end{minipage}
	\hfill
	\begin{minipage}[t]{0.48\textwidth}
		\caption{Case 2, Tilstand 1, Frekvensgraf} % Højre figurtekst og label
		\label{fig:C2T1F}
	\end{minipage}
\end{figure}

\textbf{Tilstand 2: Alle batterier leverer 0,5MW med pf 0,95 lagging.}
\begin{figure}[H]
	\centering
	\begin{minipage}[b]{0.48\textwidth}
		\centering
		\includegraphics[width=1.00\textwidth]{figurer/LargeDisturbance/Voltage2} % Venstre billede
	\end{minipage}
	\hfill
	\begin{minipage}[b]{0.48\textwidth}
		\centering
		\includegraphics[width=1.00\textwidth]{figurer/LargeDisturbance/Freq2} % Højre billede
	\end{minipage}
	\\ % Figurtekster og labels
	\begin{minipage}[t]{0.48\textwidth}
		\caption{Case 2, Tilstand 2, Spændingsgraf} % Venstre figurtekst og label
		\label{fig:C2T2V}
	\end{minipage}
	\hfill
	\begin{minipage}[t]{0.48\textwidth}
		\caption{Case 2, Tilstand 2, Frekvensgraf} % Højre figurtekst og label
		\label{fig:C2T2F}
	\end{minipage}
\end{figure}

\textbf{Tilstand 3: Alle batterier leverer 1MW med pf 0,95 lagging.}
\begin{figure}[H]
	\centering
	\begin{minipage}[b]{0.48\textwidth}
		\centering
		\includegraphics[width=1.00\textwidth]{figurer/LargeDisturbance/Voltage3} % Venstre billede
	\end{minipage}
	\hfill
	\begin{minipage}[b]{0.48\textwidth}
		\centering
		\includegraphics[width=1.00\textwidth]{figurer/LargeDisturbance/Freq3} % Højre billede
	\end{minipage}
	\\ % Figurtekster og labels
	\begin{minipage}[t]{0.48\textwidth}
		\caption{Case 2, Tilstand 3, Spændingsgraf} % Venstre figurtekst og label
		\label{fig:C2T3V}
	\end{minipage}
	\hfill
	\begin{minipage}[t]{0.48\textwidth}
		\caption{Case 2, Tilstand 3, Frekvensgraf} % Højre figurtekst og label
		\label{fig:C2T3F}
	\end{minipage}
\end{figure}

\textbf{Tilstand 4: Alle batterier leverer 1,35MW (Byerne kan betegnes som selvforsynende) med pf 0,95 lagging.}
\begin{figure}[H]
	\centering
	\begin{minipage}[b]{0.48\textwidth}
		\centering
		\includegraphics[width=1.00\textwidth]{figurer/LargeDisturbance/Voltage4} % Venstre billede
	\end{minipage}
	\hfill
	\begin{minipage}[b]{0.48\textwidth}
		\centering
		\includegraphics[width=1.00\textwidth]{figurer/LargeDisturbance/Freq4} % Højre billede
	\end{minipage}
	\\ % Figurtekster og labels
	\begin{minipage}[t]{0.48\textwidth}
		\caption{Case 2, Tilstand 4, Spændingsgraf} % Venstre figurtekst og label
		\label{fig:C2T4V}
	\end{minipage}
	\hfill
	\begin{minipage}[t]{0.48\textwidth}
		\caption{Case 2, Tilstand 4, Frekvensgraf} % Højre figurtekst og label
		\label{fig:C2T4F}
	\end{minipage}
\end{figure}

\begin{figure}[H] % (alternativt [H])
	\centering
	\includegraphics[width=1\textwidth]{figurer/SmallDisturbance/Overview}
	\caption{Overblik for spænding og effektoverførelse i nettet}
	\label{fig:C3Overview}
\end{figure}

Der observeres at når produktionen fra vindmølleparken udkobles vil der opleves et spændingsfald på både Town5 busbar - Town5 busbar er repræsentativ for alle byer - og Transmission central 60kV busbar. Det ses at spændingsfaldet i per unit er af samme størrelsesorden for både Town5 busbar og Transmission central 60kV busbar. På spændingsgraferne ses det at en stor andel af effektbidrag fra batteri vil resultere i et mindre spændingsfald ved udkobling. I tilstand 4, hvor byerne er selvforsynende, jf. \ref{fig:C3Overview}



I de 3 første tilstande observeres et spændingsfald ved Town5 busbar, når linjen 10kV Cable Town5 udkobles. Størrelsen afhænger af batteriernes effektbidrag. I tilstand 4 ses det at spændingen ved Town5 busbar stiger ved udkoblingen. Spænding ved Transmissions central 60kV busbar forbliver konstant på ca. 1pu i alle tilstande. Spændingsfaldet i tilstandende 1 - 3 er forventet, fordi tabet af den redundante linje vil øge kilde impedansen set fra Town5 busbar. Spændingsfaldet bliver mindre desto mere batteri effektbidrag pga. den reduceret strøm i transmissions og distributions kabler. Spændingsstigningen i tilstand 4 kan forklares ved at Town5 har større produktion end forbrug, derved vil den levere effekt til resten af systemet. Ved tab af 10kV Cable Town5, bliver load impedansen set fra Town5 busbar større og der vil opleves en spændingsstigning ved byens POC.

I de 4 tilstande ses det at frekvensen bliver mere stabilt ved større batteri bidrag. Stigningen på frekvensen i de første tilstande sker fordi at den samlede belastning bliver mindre da spændingen falder, produktionen forsøger at nedregulere, men kan ikke regulere hurtigt nok ift. belastningen. Dette sker ikke i tilstand 4 da spændingen ikke falder.
% !TEX root = ../SYSprojektrapport.tex
% SKAL STÅ I TOPPEN AF ALLE FILER FOR AT MASTER-filen KOMPILERES 

\section{Case 4: Den centrale batteriparks evne til at kompensere for tab af produktion}
I dette afsnit præsenteres resultater for simuleringen af case 4 iht. beskrivelsen i afsnit \ref{SimCase1}. I alle fire tilstande er spændingsændringen ved \textit{Town5 busbar} (rød linje) og \textit{Transmission central 60kV busbar} (Grøn linje) samt frekvensændringen på \textit{Transmission central 60kV busbar} (Grøn linje) præsenteret på hhv. spændingsgraf og frekvensgraf. Derudover er der lavet opsamling over spænding samt effektoverførelse andre relevante steder i systemet i tabel \ref{fig:C4Overview}. \\ \\

\textbf{Tilstand 1: Alle batterierne er frakoblet.}
\begin{figure}[H]
	\centering
	\begin{minipage}[b]{0.48\textwidth}
		\centering
		\includegraphics[width=1.00\textwidth]{figurer/LargeDisturbance/Voltage1} % Venstre billede
	\end{minipage}
	\hfill
	\begin{minipage}[b]{0.48\textwidth}
		\centering
		\includegraphics[width=1.00\textwidth]{figurer/LargeDisturbance/Freq1} % Højre billede
	\end{minipage}
	\\ % Figurtekster og labels
	\begin{minipage}[t]{0.48\textwidth}
		\caption{Case 4, Tilstand 1, Spændingsgraf} % Venstre figurtekst og label
		\label{fig:C4T1V}
	\end{minipage}
	\hfill
	\begin{minipage}[t]{0.48\textwidth}
		\caption{Case 4, Tilstand 1, Frekvensgraf} % Højre figurtekst og label
		\label{fig:C4T1F}
	\end{minipage}
\end{figure}

\textbf{Tilstand 2: Batteri central leverer 2,5MW med pf 0,95 lagging.}
\begin{figure}[H]
	\centering
	\begin{minipage}[b]{0.48\textwidth}
		\centering
		\includegraphics[width=1.00\textwidth]{figurer/LargeDisturbanceBatterypark/Voltage2} % Venstre billede
	\end{minipage}
	\hfill
	\begin{minipage}[b]{0.48\textwidth}
		\centering
		\includegraphics[width=1.00\textwidth]{figurer/LargeDisturbanceBatterypark/Freq2} % Højre billede
	\end{minipage}
	\\ % Figurtekster og labels
	\begin{minipage}[t]{0.48\textwidth}
		\caption{Case 4, Tilstand 2, Spændingsgraf} % Venstre figurtekst og label
		\label{fig:C4T2V}
	\end{minipage}
	\hfill
	\begin{minipage}[t]{0.48\textwidth}
		\caption{Case 4, Tilstand 2, Frekvensgraf} % Højre figurtekst og label
		\label{fig:C4T2F}
	\end{minipage}
\end{figure}

\textbf{Tilstand 3: Batteri central leverer 5MW med pf 0,95 lagging.}
\begin{figure}[H]
	\centering
	\begin{minipage}[b]{0.48\textwidth}
		\centering
		\includegraphics[width=1.00\textwidth]{figurer/LargeDisturbanceBatterypark/Voltage3} % Venstre billede
	\end{minipage}
	\hfill
	\begin{minipage}[b]{0.48\textwidth}
		\centering
		\includegraphics[width=1.00\textwidth]{figurer/LargeDisturbanceBatterypark/Freq3} % Højre billede
	\end{minipage}
	\\ % Figurtekster og labels
	\begin{minipage}[t]{0.48\textwidth}
		\caption{Case 4, Tilstand 3, Spændingsgraf} % Venstre figurtekst og label
		\label{fig:C4T3V}
	\end{minipage}
	\hfill
	\begin{minipage}[t]{0.48\textwidth}
		\caption{Case 4, Tilstand 3, Frekvensgraf} % Højre figurtekst og label
		\label{fig:C4T3F}
	\end{minipage}
\end{figure}

\textbf{Tilstand 4: Batterier leverer 4,41MW med pf 0,95 lagging.}
\begin{figure}[H]
	\centering
	\begin{minipage}[b]{0.48\textwidth}
		\centering
		\includegraphics[width=1.00\textwidth]{figurer/LargeDisturbanceBatterypark/Voltage4} % Venstre billede
	\end{minipage}
	\hfill
	\begin{minipage}[b]{0.48\textwidth}
		\centering
		\includegraphics[width=1.00\textwidth]{figurer/LargeDisturbanceBatterypark/Freq4} % Højre billede
	\end{minipage}
	\\ % Figurtekster og labels
	\begin{minipage}[t]{0.48\textwidth}
		\caption{Case 4, Tilstand 4, Spændingsgraf} % Venstre figurtekst og label
		\label{fig:C4T4V}
	\end{minipage}
	\hfill
	\begin{minipage}[t]{0.48\textwidth}
		\caption{Case 4, Tilstand 4, Frekvensgraf} % Højre figurtekst og label
		\label{fig:C4T4F}
	\end{minipage}
\end{figure}

\textbf{Tilstandsoverblik}
\begin{figure}[H] % (alternativt [H])
	\centering
	\includegraphics[width=1\textwidth]{figurer/LargeDisturbanceBatterypark/Overview}
	\caption{Overblik for spænding og effektoverførelse i nettet}
	\label{fig:C4Overview}
\end{figure}

I case 4 ses det at tendensen på spændingsgraferne er meget den samme som i case 4. Men ved at sammeholde tilstandsoverblikkene for case 3 og case 4 kan det ses at med en central batteripark, der levere samme effekt som 5 decentrale forsamlinger af batterier, tilsluttet systemet formår nettet at opretholde en højere spænding efter udkoblingen i tilstand 2 og 3. I tilstand 4 ses det at de decentrale batterier kan opretholde den højest spænding. Dette hænger primært sammen med at kildeimpendansen vil være højere i tilfældet med den centrale batteripark, da der er længere ud til belastningen. Det må betyde at de to typer af batteritilslutning har forskellige fordele i forskellige situation. Batteriparkens fordel i tilstand 2 og 3 er i størrelsesorden 0,01-0,03pu, hvilket kan tale for at de decentrale batterier vil være at foretrække, fordi deres fordel i tilstand 4 er noget mere betydelig. Det skal dog bemærkes at i tilstand 4 formår begge typer batterier at opretholde en tilladelig spænding indenfor de $\pm$10\% af nominel spænding.

Angående frekvensen kan der i tilstand 2 ikke observeres en forskel på de to cases. Derimod har den centrale batteripark et mindre frekvensfald i tilstand 3 og 4, hvilket betyder at dette system har den bedste frekvensstabilitet. Dette kan være pga. at den synkrone generator har en større produktion i case 4, der derved giver mere inerti i systemet og gør frekvensen mere stabil. Den ekstra produktion i forhold til case 3, er nødvendig for at kompensere for tabet i kabler og transformere fra den centrale batteripark ud til belastningerne.

\chapter{Fremtidigt arbjede}
% !TEX root = ../SYSprojektrapport.tex
% SKAL STÅ I TOPPEN AF ALLE FILER FOR AT MASTER-filen KOMPILERES 

\label{FremtidigtArbejde}

I dette projekt er der gennemført en undersøgelse af den statiske effekt batterier kan bidrage med i et elnet i forhold til stabilitetsproblemer. Det vil som fremtidigt arbejde kunne undersøges batteriers dynamiske karakteristik ved fejl, som f.eks. ved forskellige typer af kortslutninger. Her vil det være relevant at have fokus på hvordan man kan implementere regulering af batterier.

Modellen der er anvendt i projektet kunne her bruges som udgangspunkt. Det vil dog kræve at der bliver arbejdet med at implementere regulering i alle produktionsenheder, så man kan få et mere reelt billede af hvordan kontrolreserver vil interagere.

Det har i case 1 vist sig at batterier kan anvendes til fuldstændigt at forsyne en enkelt by under en fejl. Ud fra den observering kan det være relevant at undersøge hvad der er nødvendigt for at man i perioder kan køre ø-drift i en by eller et boligområde.



\chapter{Konklusion}
% !TEX root = ../SYSprojektrapport.tex
% SKAL STÅ I TOPPEN AF ALLE FILER FOR AT MASTER-filen KOMPILERES 

\label{Konklusion}
I projektet er undersøgt muligheden for at stabilisere et elnet ved implementeringen af husstandsbatterier. Dette er undersøgt med afsæt i at batterier kan udligne balancen ved flukterende produktion i et elnet. Der er fokuseret på hvordan batterier forbedrer et elnets spændings- og frekvensstabilitet i forskellige cases. Resultaterne af de forskellige cases har vist at batterier kan forbedre spændingsstabiliteten markant i alle tilfælde. Det har vist sig at batterier i nettet kan stabilisere spændingen ved fejl, samt over- og underproduktion. Dette er uafhængigt af om det er husstandsbatterier eller centrale batteriparker. Der viste sig at være mindre forskelle mellem de to batterityper i forhold spændingsstabilitet. Forskellen bunder i batteriers placering i nettet. Batteriparken giver anledning til mere tab i systemet under fejl.
\\

I forhold til frekvensstabilitet viser resultaterne at batterier kan stabilisere frekvensen ved overproduktion, da de kan kobles ind som en variabel belastning og derved udjævne belastningsforholdet i systemet.
Batterier kan også stabilisere frekvensen ved tab af produktion, men en for stor andel af produktion fra batterier vil føre til manglende inerti i systemet og derved gøre frekvensstabiliteten dårligere. Når den centrale batteripark anvendes observeres et mindre frekvensfald end med husstandsbatterier - ved samme effektbidrag - pga. synkron generatoren producerer mere for at kompensere for tab i kabler. Derved har systemet mere inerti.
\\

Det er for projektet værd at bemærke at simplificering af det danske elnet kan resultere i nogle misvisninger, da der typisk vil indgå roterende belastningen samt synkron kondensere, der alle vil bidrage til forbedre frekvensstabilitet. Derudover er al central produktion samlet i to enheder; synkron generatoren og vindmølleparken, der gør systemets selektivititet mindre. Samtidig er der udført simuleringer hvor batterier udgør en stor del af produktionen i systemet, hvilket er urealistisk, men kan illustrere tendenser ved anvendelse batterier. \\
Der er i modellen heller ikke taget højde for udlandsforbidnelser. Generelt viser projektet at en større mængde batterier i et elnet vil kunne stabilisere både frekvens og spænding ved anvendelse som supplerende produktionskilde eller variabel belastning.




\printbibliography

\end{document}


