% !TEX root = ../SYSprojektrapport.tex
% SKAL STÅ I TOPPEN AF ALLE FILER FOR AT MASTER-filen KOMPILERES 

\label{Afgraensning}

Projektet afgrænses til at skal indeholde en undersøgelse af følgende fire cases:

\begin{description}
	\item[Case 1] Husstandsbatteriers evne til at stabilisere elnettet ved fejl på nettet.
	\item[Case 2] Husstandsbatteriers evne til at absorbere overproduktion.
	\item[Case 3] Husstandsbatteriers evne til at kompensere for tab af produktion.
	\item[Case 4] Husstandsbatteriers stabilierende effekt af elnettet kontra en central batteripark.
\end{description}	
	
Derudover kan følgende tre cases blive en del af projektet, hvis tiden til det forefindes. Hvis de tre cases ikke bliver en del af projektet vil det være relevante cases at undersøge i et opfølgende projekt.

\begin{description}
	\item[Case 4] Husstandsbatteriers stabilierende effekt af elnettet kontra en central batteripark.
	\item[Case 5] Ø-drift af et boligområde.
	\item[Case 6] Husstandsbatteriers evne til at bidrage med kortslutningseffekt.
\end{description}

En beskrivelse af de cases, der udføres i projektet er lavet i kapitel \ref{Simulering}.

Det kunne være relevant at lave en business case på hvordan implementering af batterier i elnettet kunne udføres optimalt. Men i dette projekt fokuseres på de tekniske fordele det kunne medføre at implementere batterier i elnettet. Derfor vil en business case være endnu en ting der vil være relevant at lave i et opfølgende projekt.

Det vil i projektet have været relevant at undersøge hvordan man kan implementere regulering i de forskellige produktionsenheder i simuleringen af de forskellige cases, men dette er for simplificerings skyld ikke en del af projektet.