% !TEX root = ../SYSprojektrapport.tex
% SKAL STÅ I TOPPEN AF ALLE FILER FOR AT MASTER-filen KOMPILERES 

\label{Indledning}

Denne rapport er udarbejdet i forbindelse med kurset Energy System stability på Aarhus universitet. 
Danmark har et af verdens mest stabile energiforsyningerne med gode forbindelse til omkringliggende lande. Men pga. de høje ambitioner om at nedsætte CO2 udslippet i hele Europa ændres energiproduktionerne i stor grad til vedvarende energikilder. Da de vedvarende energikilder afhængig af vejrforholdene bliver produktionen mere fluktuerende, det kan skabe ubalancen mellem produktion og forbrug og derved give anledning til stabilitetsproblemer. En af mulighederne for at undgå stabilitetsproblemerne er ved at implementer batterier i el nettet.

I dette projekt undersøges hvordan husstandsbatterier kan stabilisere et el net. Det undersøges hvilke former for stabilitetsproblemer batterier kan afhjælpe og derefter opbygges et simplificeret el net i PowerFactory for at verificere de teoretiske undersøgelser. Tilslut dokumenteres resultater og der diskuteres på de enkelte undersøgelser.

