% !TEX root = ../SYSprojektrapport.tex
% SKAL STÅ I TOPPEN AF ALLE FILER FOR AT MASTER-filen KOMPILERES 

\label{Konklusion}
I projektet er undersøgt muligheden for at stabilisere et elnet ved implementeringen af husstandsbatterier. Dette er undersøgt med afsæt i at batterier kan udligne balancen ved flukterende produktion i et elnet. Der er fokuseret på hvordan batterier forbedrer et elnets spændings- og frekvensstabilitet i forskellige cases. Resultaterne af de forskellige cases har vist at batterier kan forbedre spændingsstabiliteten markant i alle tilfælde. Det har vist sig at batterier i nettet kan stabilisere spændingen ved fejl, samt over- og underproduktion. Dette er uafhængigt af om det er husstandsbatterier eller centrale batteriparker. Der viste sig at være mindre forskelle mellem de to batterityper i forhold spændingsstabilitet. Forskellen bunder i batteriers placering i nettet. Batteriparken giver anledning til mere tab i systemet under fejl.
\\

I forhold til frekvensstabilitet viser resultaterne at batterier kan stabilisere frekvensen ved overproduktion, da de kan kobles ind som en variabel belastning og derved udjævne belastningsforholdet i systemet.
Batterier kan også stabilisere frekvensen ved tab af produktion, men en for stor andel af produktion fra batterier vil føre til manglende inerti i systemet og derved gøre frekvensstabiliteten dårligere. Når den centrale batteripark anvendes observeres et mindre frekvensfald end med husstandsbatterier - ved samme effektbidrag - pga. synkron generatoren producerer mere for at kompensere for tab i kabler. Derved har systemet mere inerti.
\\

Det er for projektet værd at bemærke at simplificering af det danske elnet kan resultere i nogle misvisninger, da der typisk vil indgå roterende belastningen samt synkron kondensere, der alle vil bidrage til forbedre frekvensstabilitet. Derudover er al central produktion samlet i to enheder; synkron generatoren og vindmølleparken, der gør systemets selektivititet mindre. Samtidig er der udført simuleringer hvor batterier udgør en stor del af produktionen i systemet, hvilket er urealistisk, men kan illustrere tendenser ved anvendelse batterier. \\
Der er i modellen heller ikke taget højde for udlandsforbidnelser. Generelt viser projektet at en større mængde batterier i et elnet vil kunne stabilisere både frekvens og spænding ved anvendelse som supplerende produktionskilde eller variabel belastning.

