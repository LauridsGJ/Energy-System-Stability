% !TEX root = ../SYSprojektrapport.tex
% SKAL STÅ I TOPPEN AF ALLE FILER FOR AT MASTER-filen KOMPILERES 


\label{Modelopbygning}

\section{Model}

For at undersøge husstandsbatterier indflydelse i et el net er der simuleret 5 byer med ca. 2000 hustande, der alle har et batteri installeret. Hver hustand er sat til at have et gennemsnitligt dagligt forbrug på 14kWh,\footnote{https://orsted.dk/Privat/Faa-en-lavere-regning/Kom-godt-i-gang-og-spar-paa-energien/Test-dit-gennemsnitsforbrug/Elforbrug}. Det giver et gennemsnitlig forbrug på 1183kW per by, og en total belastning på 5914kW. For at finde en maks. belastning samt der taget udgangspunkt i Energinets belastning og produktions information fra Danmark. Det er undersøgt hvor stor en del den gennemsnitlige belastning i modelen udgør af Danmarks gennemsnitlige belastning og derved fundet en skaleringsfaktor på 716. Denne skallerings faktor er brugt til at finde den samlede maks. belastning for byerne ved dividere Danmarks maks. belastning med faktoren. Den maks. belastning for hver by ligger derved på 1524kW. 
På figur \ref{fig:Simdis} ses distributionsnettets opbygning der er lavet som en ringforbindelse fra Distribution busbar, med en ekstra tværgående linje. Ved hver by er placeret en transformer der transformere spændingen fra 10kV til 0,4kV. Hver by er simplificeret til en belastning, et batteri og et solcelle anlæg. Den maksimale batteri kapacitet er fundet i forhold til at hver hustand har monteret en Tesla powerwall, der kan levere 5kW, så hver by har en batteri kapacitet på 10000kW. Den maksimale solcelle produktionsmængde er fundet ud fra Energinets oplysninger for hele landet og derefter divideret med skaleringsfaktoren, som giver en maksimal produktion på 912kW per by. 

 
 \begin{figure}[H] % (alternativt [H])
 	\centering
 	\includegraphics[width=1\textwidth]{figurer/Sim_model_2}
 	\caption{Systemets belastning og distribution}
 	\label{fig:Simdis}
 \end{figure}
    

På figur \ref{fig:SimTrans} ses transmissionsnettet. For at simplificer modelen er alt vind- og fossilt energi samlet i to generator. Størrelsen af vind generatoren er valgt ud fra den gennemsnitlige produktion af vindenergi i Danmark. Vindmølle generatoren producere derved 2MW. Den fossile generator er sat som referencemaskine og ændre sin produktion i forhold til belastningen i nettet. Produktions enhederne transformeres først op til et 150kV transmissions net og derefter til 60kV. 60kV Transmissionen er lavet med 2 redundante kabler så det kan undersøges hvad der sker, hvis det ene kable falder ud. Derefter transformeres spændingen ned til distributions niveau. Derudover er der placeret en stor batteri enhed på 60kV busbaren til at kunne undersøge forskellen mellem husstandsbatterier og et centralt placeret batteri.

Generatorer, kabler og transformere er lavet som simplificeret modeler med generelle værdier for de enkelte komponenter tilpasset spænding og belastning i de forskellige niveauer.
 

\begin{figure}[H] % (alternativt [H])
	\centering
	\includegraphics[width=1\textwidth]{figurer/Sim_model_1}
	\caption{systemets produktion transmission}
	\label{fig:SimTrans}
\end{figure}
