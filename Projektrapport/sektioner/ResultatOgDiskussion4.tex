% !TEX root = ../SYSprojektrapport.tex
% SKAL STÅ I TOPPEN AF ALLE FILER FOR AT MASTER-filen KOMPILERES 

\section{Case 4: Den centrale batteriparks evne til at kompensere for tab af produktion}
I dette afsnit præsenteres resultater for simuleringen af case 4 iht. beskrivelsen i afsnit \ref{SimCase1}. I alle fire tilstande er spændingsændringen ved \textit{Town5 busbar} (rød linje) og \textit{Transmission central 60kV busbar} (Grøn linje) samt frekvensændringen på \textit{Transmission central 60kV busbar} (Grøn linje) præsenteret på hhv. spændingsgraf og frekvensgraf. Derudover er der lavet opsamling over spænding samt effektoverførelse andre relevante steder i systemet i tabel \ref{fig:C4Overview}. \\ \\

\textbf{Tilstand 1: Alle batterierne er frakoblet.}
\begin{figure}[H]
	\centering
	\begin{minipage}[b]{0.48\textwidth}
		\centering
		\includegraphics[width=1.00\textwidth]{figurer/LargeDisturbance/Voltage1} % Venstre billede
	\end{minipage}
	\hfill
	\begin{minipage}[b]{0.48\textwidth}
		\centering
		\includegraphics[width=1.00\textwidth]{figurer/LargeDisturbance/Freq1} % Højre billede
	\end{minipage}
	\\ % Figurtekster og labels
	\begin{minipage}[t]{0.48\textwidth}
		\caption{Case 4, Tilstand 1, Spændingsgraf} % Venstre figurtekst og label
		\label{fig:C4T1V}
	\end{minipage}
	\hfill
	\begin{minipage}[t]{0.48\textwidth}
		\caption{Case 4, Tilstand 1, Frekvensgraf} % Højre figurtekst og label
		\label{fig:C4T1F}
	\end{minipage}
\end{figure}

\textbf{Tilstand 2: Batteri central leverer 2,5MW med pf 0,95 lagging.}
\begin{figure}[H]
	\centering
	\begin{minipage}[b]{0.48\textwidth}
		\centering
		\includegraphics[width=1.00\textwidth]{figurer/LargeDisturbanceBatterypark/Voltage2} % Venstre billede
	\end{minipage}
	\hfill
	\begin{minipage}[b]{0.48\textwidth}
		\centering
		\includegraphics[width=1.00\textwidth]{figurer/LargeDisturbanceBatterypark/Freq2} % Højre billede
	\end{minipage}
	\\ % Figurtekster og labels
	\begin{minipage}[t]{0.48\textwidth}
		\caption{Case 4, Tilstand 2, Spændingsgraf} % Venstre figurtekst og label
		\label{fig:C4T2V}
	\end{minipage}
	\hfill
	\begin{minipage}[t]{0.48\textwidth}
		\caption{Case 4, Tilstand 2, Frekvensgraf} % Højre figurtekst og label
		\label{fig:C4T2F}
	\end{minipage}
\end{figure}

\textbf{Tilstand 3: Batteri central leverer 5MW med pf 0,95 lagging.}
\begin{figure}[H]
	\centering
	\begin{minipage}[b]{0.48\textwidth}
		\centering
		\includegraphics[width=1.00\textwidth]{figurer/LargeDisturbanceBatterypark/Voltage3} % Venstre billede
	\end{minipage}
	\hfill
	\begin{minipage}[b]{0.48\textwidth}
		\centering
		\includegraphics[width=1.00\textwidth]{figurer/LargeDisturbanceBatterypark/Freq3} % Højre billede
	\end{minipage}
	\\ % Figurtekster og labels
	\begin{minipage}[t]{0.48\textwidth}
		\caption{Case 4, Tilstand 3, Spændingsgraf} % Venstre figurtekst og label
		\label{fig:C4T3V}
	\end{minipage}
	\hfill
	\begin{minipage}[t]{0.48\textwidth}
		\caption{Case 4, Tilstand 3, Frekvensgraf} % Højre figurtekst og label
		\label{fig:C4T3F}
	\end{minipage}
\end{figure}

\textbf{Tilstand 4: Batterier leverer 4,41MW med pf 0,95 lagging.}
\begin{figure}[H]
	\centering
	\begin{minipage}[b]{0.48\textwidth}
		\centering
		\includegraphics[width=1.00\textwidth]{figurer/LargeDisturbanceBatterypark/Voltage4} % Venstre billede
	\end{minipage}
	\hfill
	\begin{minipage}[b]{0.48\textwidth}
		\centering
		\includegraphics[width=1.00\textwidth]{figurer/LargeDisturbanceBatterypark/Freq4} % Højre billede
	\end{minipage}
	\\ % Figurtekster og labels
	\begin{minipage}[t]{0.48\textwidth}
		\caption{Case 4, Tilstand 4, Spændingsgraf} % Venstre figurtekst og label
		\label{fig:C4T4V}
	\end{minipage}
	\hfill
	\begin{minipage}[t]{0.48\textwidth}
		\caption{Case 4, Tilstand 4, Frekvensgraf} % Højre figurtekst og label
		\label{fig:C4T4F}
	\end{minipage}
\end{figure}

\textbf{Tilstandsoverblik}
\begin{figure}[H] % (alternativt [H])
	\centering
	\includegraphics[width=1\textwidth]{figurer/LargeDisturbanceBatterypark/Overview}
	\caption{Overblik for spænding og effektoverførelse i nettet}
	\label{fig:C4Overview}
\end{figure}

I case 4 ses det at tendensen på spændingsgraferne er meget den samme som i case 4. Men ved at sammeholde tilstandsoverblikkene for case 3 og case 4 kan det ses at med en central batteripark, der levere samme effekt som 5 decentrale forsamlinger af batterier, tilsluttet systemet formår nettet at opretholde en højere spænding efter udkoblingen i tilstand 2 og 3. I tilstand 4 ses det at de decentrale batterier kan opretholde den højest spænding. Dette hænger primært sammen med at kildeimpendansen vil være højere i tilfældet med den centrale batteripark, da der er længere ud til belastningen. Det må betyde at de to typer af batteritilslutning har forskellige fordele i forskellige situation. Batteriparkens fordel i tilstand 2 og 3 er i størrelsesorden 0,01-0,03pu, hvilket kan tale for at de decentrale batterier vil være at foretrække, fordi deres fordel i tilstand 4 er noget mere betydelig. Det skal dog bemærkes at i tilstand 4 formår begge typer batterier at opretholde en tilladelig spænding indenfor de $\pm$10\% af nominel spænding.

Angående frekvensen kan der i tilstand 2 ikke observeres en forskel på de to cases. Derimod har den centrale batteripark et mindre frekvensfald i tilstand 3 og 4, hvilket betyder at dette system har den bedste frekvensstabilitet. Dette kan være pga. at den synkrone generator har en større produktion i case 4, der derved giver mere inerti i systemet og gør frekvensen mere stabil. Den ekstra produktion i forhold til case 3, er nødvendig for at kompensere for tabet i kabler og transformere fra den centrale batteripark ud til belastningerne.