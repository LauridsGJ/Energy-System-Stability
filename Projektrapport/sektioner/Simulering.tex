% !TEX root = ../SYSprojektrapport.tex
% SKAL STÅ I TOPPEN AF ALLE FILER FOR AT MASTER-filen KOMPILERES 

\label{Simulering}

Der er i case simuleringerne som beskrevet i kapitel \ref{Afgraensning} ikke implementeret nogen form for regulering i produktionsenhederne, dvs. vindmølleparken, synkron generatoren og batterierne. Til gengæld anvendes en meget stor inertikonstant på 100MW*s for synkron generatoren, for at tilføre systemet en mere realistisk dæmpning i forhold til frekvensændringer.

\section{Case 1: Husstandsbatteriers evne til at stabilisere elnettet ved fejl på nettet}
\label{SimCase1}
For at simulere case 1 anvendes et scenarie, hvor systemet er fuldt belastet, dvs. hver by trækker 1,524MW med pf 0,95 lagging. Efter 10s mister systemet kablet 1\textit{0kV Cable Town5} ved en udkobling. Dette betegnes som en small disturbance fejl der vil skabe short term ubalance i powerflowet.\\
Vindmølleparken generer 2MW med pf 0,99 lagging, solcellerne i hver by levere 0,15MW med pf 0.8 lagging og synkron generatoren er reference maskine.\\
Case 1 simuleres i fire forskellige tilstande.

\begin{description}
	\item[Tilstand 1] Alle batterierne er frakoblet.
	\item[Tilstand 2] Batteriet i \textit{Town4} leverer 0,25MW og batteriet i \textit{Town5} leverer 0,5MW. Begge med pf 0,95 lagging.
	\item[Tilstand 3] Batteriet i \textit{Town4} leverer 0,5MW og batteriet i \textit{Town5} leverer 1MW. Begge med pf 0,95 lagging.
	\item[Tilstand 4] Batteriet i \textit{Town4} leverer 0,75MW og batteriet i \textit{Town5} leverer 1,5MW. Begge med pf 0,95 lagging.
\end{description}

Parametrene der overvåges er for synkron generatoren P, Q og V.
For batterierne i \textit{Town4} og \textit{Town5} P og Q. Samt V for \textit{Transmission 60kV busbar}, \textit{Town1 busbar}, \textit{Town2 busbar}, \textit{Town3 busbar}, \textit{Town4 busbar} og \textit{Town5 busbar}.
Derudover overvåges systemfrekvensen.

\section{Case 2: Husstandsbatteriers evne til at absorbere overproduktion}
\label{SimCase2}
For at simulere case 2 anvendes et scenarie, hvor systemet er fuldt belastet, dvs. hver by trækker 1,524MW med pf 0,95 lagging. Efter 10s mister systemet \textit{Town2} pga. en udkobling ved Distribution by2 busbar. Dette betegnes som en large disturbance fejl, der kunne udvikle sig til en long term ubalance i systemet. I case 2 simuleres kun den short term påvirkning udkoblingen vil have på systemet, da det stadigvæk vil give et indblik i batteriernes evne til at absorbere overproduktion.\\
Vindmølleparken generer 2MW med pf 0,95 lagging, solcellerne i hver by levere 0,15MW med pf 0.8 lagging og synkron generatoren er reference maskine.\\
Case 2 simuleres i to forskellige tilstande.

\begin{description}
	\item[Tilstand 1] Alle batterierne er frakoblet.
	\item[Tilstand 2] Batterierne i alle byer kobles ind 0,5s efter fejlen og absorberer 0,304MW som kompensation for tabet af byen. Alle med pf 0,95 lagging.
\end{description}

Parametrene der overvåges er for synkron generatoren P, Q og V.
For batterierne i alle 5 byer P og Q. Samt V for \textit{Transmission 60kV busbar}, \textit{Town1 busbar}, \textit{Town2 busbar}, \textit{Town3 busbar}, \textit{Town4 busbar} og \textit{Town5 busbar}.
Derudover overvåges systemfrekvensen.

\section{Case 3: Husstandsbatteriers evne til at kompensere for tab af produktion}
\label{SimCase3}
For at simulere case 3 anvendes et scenarie, hvor systemet er fuldt belastet, dvs. hver by trækker 1,524MW med pf 0,95 lagging, samt at kablet \textit{Transmission cable2} er koblet ud pga. vedligeholdelse. Efter 10s mister systemet vindmølleparken pga. en udkobling ved \textit{Transmission central 150kV busbar}. Dette betegnes som en large disturbance fejl, der kunne udvikle sig til en long term ubalance eller muligvis blackout i større dele af systemet. I case 3 simuleres kun den short term påvirkning det vil have på systemet, da det stadigvæk vil give et indblik i batteriernes evne til at kompensere for mistet produktion.\\
Vindmølleparken generer 2MW med pf 0,95 lagging, solcellerne i hver by levere 0,15MW med pf 0.8 lagging og synkron generatoren er reference maskine.\\
Case 3 simuleres i fire forskellige tilstande.

\begin{description}
	\item[Tilstand 1] Alle batterierne er frakoblet.
	\item[Tilstand 2] Alle batterier leverer 0,5MW med pf 0,95 lagging.
	\item[Tilstand 3] Alle batterier leverer 1MW med pf 0,95 lagging.
	\item[Tilstand 4] Alle batterier leverer 1,35MW (Byerne kan betegnes som selvforsynende) med pf 0,95 lagging.
\end{description}

Parametrene der overvåges er for synkron generatoren P, Q og V.
For batterierne i alle 5 byer P og Q. Samt V for \textit{Transmission 60kV busbar}, \textit{Town1 busbar}, \textit{Town2 busbar}, \textit{Town3 busbar}, \textit{Town4 busbar} og \textit{Town5 busbar}.
Derudover overvåges systemfrekvensen.

\section{Case 4: Den centrale batteriparks evne til at kompensere for tab af produktion}
\label{SimCase4}
Case 4 er den samme som case 3, bortset fra at batteriparken anvendes i stedet for husstandsbatterierne.
Case 4 simuleres i fire forskellige tilstande.

\begin{description}
	\item[Tilstand 1] Batteriparken er frakoblet.
	\item[Tilstand 2] Batteriparken leverer 2,5MW med pf 0,95 lagging.
	\item[Tilstand 3] Batteriparken leverer 5MW med pf 0,95 lagging.
	\item[Tilstand 4] Batteriparken leverer 6,75MW med pf 0,95 lagging.
\end{description}

Parametrene der overvåges er for synkron generatoren P, Q og V.
For batteriparken P og Q. Samt V for \textit{Transmission 60kV busbar}, \textit{Town1 busbar}, \textit{Town2 busbar}, \textit{Town3 busbar}, \textit{Town4 busbar} og \textit{Town5 busbar}.
Derudover overvåges systemfrekvensen.
